\documentclass[../Master.tex]{subfiles}
\begin{document}
\begin{multicols*}{2}
	\section{X-Ray Diffraction}
	\subsection{Diffracted intensity}
	\subsubsection{Monophasic Sample}
	The intensity of diffraction for a generic hkl plane is:
	\[
		I_{hkl} = K_{e} K_{hkl} \frac{1}{2 \mu}
	\]
	\subsubsection{Polyphasic Sample}
	The intensity of diffraction for a generic hkl plane is:
	\[
		I_{hkl, \ \alpha} = K_{e} K_{hkl, \alpha } \cdot w_{\alpha} \frac{1}{\mu _{m}^{*}  }
	\]
	\subsection{Internal Standard}
	A standard must be added to the sample:
	\[
		\frac{I_{hkl~\alpha}}{I_{hkl~\beta} } = K \cdot \frac{w_{\alpha}}{w_{\beta}}
	\]
	Using a calibration curve one can obtain:
	\begin{gather*}
		\frac{I_{hkl~\alpha}}{I_{hkl~\beta} } = K\cdot \frac{w_{\alpha}}{w_{\beta}} \\
		w_{\alpha} = \frac{I_{\alpha,hkl}}{I_{std,hkl}} \cdot \frac{w_{std}}{K_{\alpha,std} } \\
		w_{\alpha, ini} = \frac{w_{\alpha}}{1-w_{std}}
	\end{gather*}
	\subsection{Relative Intensity Ration Method}
	The same concept as internal standard is applied using corindone. The 113 coridone peak and the highest intensity	\( \alpha  \) peak are considered. \\
	The RIR value can be found in databases like PDF.
	\[
		w_{\alpha} = \frac{I_{\alpha,hkl}}{I_{cor,113}} \cdot \frac{w_{std}}{RIR_{\alpha ,COR}  }
	\]
	\subsubsection{Generalized RIR}
	It is possibile to use different corindone and alpha peak, also wit lower relative intensity.
	\[
		w_{\alpha} = \frac{I_{\alpha,hkl}}{I_{cor,hkl}} \cdot \frac{I_{cor,hkl}^{REL} }{I_{\alpha, hkl}^{REL}  }\cdot  \frac{w_{std}}{RIR_{\alpha,cor} }
	\]
	\subsubsection{Normalized RIR - Chung Equation}
	\begin{gather*}
		w_{i} =\left[ \left( \frac{RIR_{i}}{I_{i} } \right) \cdot \sum_{i=1} ^{n} \left( \frac{I_{i}}{RIR_{i} } \right)\right]^{-1} \\
		I_{i}= \frac{I_{line,i}}{I_{i, REL}} \\
		\text{if } \sum_{i=1}^{n} w_{i} = 1
	\end{gather*}
	\subsubsection{Amorphous determination}
	A standard is added to the sample.\\
	Using the normalization condition from the Chung equation it possibile to verify the presence of an amorphous phase:
	\[
		\sum_{i=1} ^{n} \left( \frac{I_{i}}{RIR_{i} } \right) >,=,< I_{cor}\left[ \frac{1-w_{cor} }{w_{cor} } \right]
	\]
	\begin{itemize}
		\item if > something is wrong
		\item if = there is no amorphous phase
		\item if < thers is an amorphous phase
	\end{itemize}
	If the amorphous is present the weight fraction of the other componente must be corrected:
	\[
		w_{\text{i, corrected}}  = \frac{w_{\text{std, real}} }{w_{\text{std, calculated}} } \cdot w_{i}
	\]
	The amorphous can be quantified using:
	\[
		w_{\text{amorphous}} = 1 - \sum_{i=1}^{N} w_{\text{i, corrected}}
	\]
	\subsection{Thermal Gravimetic Analysis}
	It's possibile to calculate the n multiple of the MW of the mass decrease/increase during a thermal event.
	\begin{gather*}
		A \to B + R \\
		nPM(\text{R}) = \frac{m(\text{R})}{m(\text{A}) } \cdot  MW(\text{A})
	\end{gather*}
\end{multicols*}
\end{document}
