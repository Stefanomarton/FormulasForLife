\documentclass[Master.tex]{subfiles}
\begin{document}
\begin{multicols}{2}
		 \section{Reaction rate}
		  Is always true that:
		  \[
				   r = \frac{dx}{dt} = \frac{1}{\nu _{B} } \frac{dCb}{dt}
		  \]
		 \section{First order reactions}
		  \begin{gather*}
				   A \to P \\
				   r = -\frac{dC_{A}}{dt} = k[A]\\
				   - \ln C_{A} + \ln C_{0A} = kt\\
				   \frac{C_{A} }{C_{A}^{0} } = e^{-kt}
		  \end{gather*}

		  \subsection{Half-life time}
				   \[
						    t_{\frac{1}{2}} = \frac{\ln2}{k} \quad [s^{-}]
				   \]

		  \subsection{Life time}

		 \section{Second order reactions}
		  \subsection{Unimolecolar reaction}
				   \begin{gather*}
						    r = \frac{dC_{A}}{dt} = k[A]^{2} \\
						    \frac{1}{C_{A} } - \frac{1}{C_{A} ^{0}} = kt
				   \end{gather*}
				   \subsubsection{Half-life time}
						    \[
								     t_{\frac{1}{2}} = \frac{1}{C_{A} ^{0}k}
						    \]

		  \subsection{Bimolecolar reaction}
				   \begin{gather*}
						    r = - \frac{dC_{A}}{dt} = - \frac{dC_{B}}{dt} = \frac{dx}{dt} = k[A][B] \\
						    \frac{1}{C_{A} ^{0} - C_{B} ^{0} }~\ln\left[\frac{(C_{A} ^{0} - x)C_{B} ^{0} }{(C_{B} ^{0} - x) C_{A} ^{0} }\right] = kt\\
						    \frac{C_{B}}{C_{A}} = \frac{C_{B} ^{0} }{C_{A} ^{0} }e^{(C_{A} ^{0}-C_{B} ^{0}  )kt}
				   \end{gather*}

				   \subsubsection{Half-life time}
						    \[
								     t_{\frac{1}{2}} = \frac{1}{C_{A} ^{0} - C_{B} ^{0} k}~ ~  \ln\left(  \frac{2C_{A} ^{0} - C_{B}^{0} }{C_{A}^{0}} \right)
						    \]

		 \section{Zero order reactions}
		  \begin{gather*}
				   r = k\\
				   C = C^0 - kt
		  \end{gather*}
		  \subsection{Half-life time}
				   \[
						    t_{\frac{1}{2}}  = \frac{1}{2}t_{f}
				   \]
		 \section{nth order reactions}
		  \begin{gather*}
				   r = k[A]^{n} \\
				   \frac{1}{C^{n-1} } - \frac{1}{C^{0(n-1 )} } = (n-1)kt\\
		  \end{gather*}
		  \subsection{Half-life time}
				   \[
						    t_{\frac{1}{2}} = \frac{2^{n-1} -1 }{(C^{0} )^{n-1} k(n-1)}
				   \]

		 \section{Parallel reactions}
		  In this condition A react with different velocity constant towards different
		  products W,V and U.
		  \begin{gather*}
				   \ce{ A ->[{k_{1}}] U} \\
				   \ce{ A ->[{k_{2}}] W} \\
				   \ce{ A ->[{k_{3}}] V} \\
				   r =  - \frac{dC_{A} }{dt} = k_{1}C_{A} + k_{2}C_{A} + k_{3}C_{A} = kC_{A}
		  \end{gather*}

		  Equation for the products can be written as:
		  \begin{gather*}
				   C_{U} - C_{U}^{0} = \frac{k_{1}C_{A}^{0}}{k}(1-e^{-kt})\\
				   C_{V} - C_{V}^{0} = \frac{k_{2}C_{A}^{0}}{k}(1-e^{-kt})\\
				   C_{W} - C_{W}^{0} = \frac{k_{3}C_{A}^{0}}{k}(1-e^{-kt})
		  \end{gather*}
		  k can be determined sperimentally from the relation between \( C_{A}  \) and time, in this case it would be a first order relation.
		  Concentration of the products can be determined with sperimental methods. At this point the following system can be used:

		  \begin{equation*}
				   \begin{cases}
						    k_{1} + k_{2} + k_{3} = k                  \\
						    \frac{C_{U}}{C_{W} } = \frac{k_{1}}{k_{2}} \\
						    \frac{C_{U} }{C_{V}} = \frac{k_{2}}{k_{3}}
				   \end{cases}
		  \end{equation*}

		 \section{Consecutive reactions}
		  \begin{gather*}
				   \ce{ A ->[{k_{1}}] B ->[{k_{2}}]C}\\
		  \end{gather*}
		  This is the simpliest case, with all first order reactions
		  \begin{gather*}
				   r = -\frac{dC_{A}}{dt} = k_{1}C_{A}\\
				   r = \frac{dC_{B}}{dt} = k_{1}C_{A} - k_{2}C_{B}\\
				   r = \frac{dC_{C}}{dt} = k_{2}C_{B}
		  \end{gather*}
		  The $C_{C}$ value in relation to time, in this case would be:
		  \[
				   C_{C} = C_{A}^{0}\left[1 -\frac{k_{2}e^{-k_{1}t} -k_{1}e^{-k_{2}t}}{k_{2}-k_{1}} \right] + C_{B}^{0}(1-e^{-k_{2}t})+C_{C}^{0}
		  \]
		  In case that \( C_{B}^0=0, C_{C}^0 = 0 \) then:
		  \[
				   C_{C} = C_{A}^{0}\left[1 -\frac{k_{2}e^{-k_{1}t} -k_{1}e^{-k_{2}t}}{k_{2}-k_{1}} \right]
		  \]
		 \section{Opposite reactions}

		 \section{Arrhenius's equation}
		  \begin{gather*}
				   k = A \exp \left( -\frac{Ea}{RT} \right)\\
				   \ln k = \ln A - \frac{E_{a}}{RT}
		  \end{gather*}
\end{multicols}
\end{document}
