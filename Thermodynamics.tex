\documentclass[Master.tex]{subfiles}
\begin{document}
\begin{multicols}{2}

		 \section{Perfect gases}
		  \subsection{State equation}
				   \begin{gather*}
						    pV = nRT\\
						    pV_{m} = RT
				   \end{gather*}
		  \subsection{Properties}
				   \subsubsection{Molar mass}
						    $$ d_{gas} \cdot V_{m} = MW $$
				   \subsubsection{Boyle's law}
						    $$ p_1 V_1 = p_2 V_2 \quad T = const$$
				   \subsubsection{Charle's law}
						    $$ \frac{V_1}{T_1} = \frac{V_2}{T_2} \quad p = const $$
				   \subsubsection{Gay-Lussac's Law}
						    $$ \frac{p_1}{T_1} = \frac{p_2}{T_2} \quad V = const $$

		  \subsection{Compression factor}
				   $$ Z = \frac{PV}{nRT} $$
				   Z = 1 for perfect gases
		 \section{Real gases}
		  \subsection{Virial equation}

				   \[
						    pV_{m}=RT\left( 1+\frac{B}{V_{m}}+\frac{C}{V^{2}_{m}}+\dots \right)
				   \]

		  \subsection{van der Waals equation}
				   \begin{gather*}
						    p = \frac{nRT}{V-nb} - a \left(\frac{n}{V} \right) ^{2}  \\
						    p = \frac{nRT}{V_{m} - b } - \frac{a}{V_{m}^2}
				   \end{gather*}
		  \subsection{Compression factor}
				   \begin{gather*}
						    Z = \frac{V_{m}}{V_{m}-b} - \frac{a}{RTV_{m}} \\
						    \text{if } y = \frac{1}{V_{m} } \implies \frac{1}{1-by} - \frac{ay}{RT}
				   \end{gather*}
		  \subsection{Boyle Temperature}
				   \begin{gather*}
						    \lim_{p \to 0}  \frac{\mathrm{dZ}}{\mathrm{dp}} = 0 \quad p \to 0 \implies \frac{1}{V_{m}} \to 0 \\
						    y =\frac{1}{V_{m} }\quad  \lim_{y \to 0} \frac{1}{1-by} - \frac{ay}{RT} = \\
						    T_{b} = \frac{a}{Rb}
				   \end{gather*}
		  \subsection{Critical and reduced variables}
				   \[
						    T_{c} = \frac{8a}{27Rb} \quad p_{c} = \frac{a}{27b^{2} } \quad V_{c} = 3b \quad Z_{c} = \frac{p_{c} V_{c}}{RT_{c}} \\
				   \]

				   The van der Waals equation can be rewritten :
				   \[
						    p_{r} = \frac{8T_{r} }{3V_{r} - 1 } - \frac{3}{V^{2} _{r} }
				   \]

		 \section{Internal Energy}
		  \begin{gather*}
				   \Delta U = q + w\\
				   dU = \delta w + \delta q \\
				   dU = TdS - pdV \\
				   dU = \left( \frac{\partial U}{\partial S}\right)_{V}dS + \left( \frac{\partial U}{\partial V} \right)_{S} dV \\
		  \end{gather*}
		  Is always true that:
		  \begin{gather*}
				   dU = \int_{T_1}^{T_2} C_{V}dT  \\
				   \Delta U = nC_{v,m}\Delta T
		  \end{gather*}

		 \section{Enthalpy}
		  \begin{gather*}
				   H = U + pV\\
				   \Delta H = \Delta U + \Delta pV \\
				   dH = -SdT + Vdp
		  \end{gather*}
		  For the perfect gas is also true that:
		  \[
				   \Delta H = \Delta U + \Delta (nRT)
		  \]

		  \subsection{Temperature dependence}
				   \subsubsection{Pure substance}
						    \[
								     \Delta H(T_2) = \Delta H(T_1) + \int _{T_1} ^{T_2} C_{p}dT\\
								     .\]
				   \subsubsection{Chemical reaction or phase transition}
						    \begin{gather*}
								     \Delta H(T_2) = \Delta H_{r}T_1 + \Delta C_{p}(T_2 - T_1)\\
								     \Delta C_{p} = [c C_{p,C} + d C_{p,D}] - [a C_{p,A} + b C_{p,B}] \\
						    \end{gather*}

		 \section{Entropy}

		  \begin{gather*}
				   \Delta S_{TOT} \geq 0 \\
				   \Delta S  = \frac{q}{T} \implies \Delta S = \int _{i} ^{f} \frac{q_{rev}}{T}
		  \end{gather*}
		  For an ideal gas it is always true that:
		  \[
				   \Delta S = nC_{m}\ln \left( \frac{T_{f}}{T_{i} }  \right) + nR\ln \left( \frac{V_{f}}{V_{i} }  \right)
		  \]
		  Only mixing two gases results in:
		  \[
				   \Delta S_{mix} = -R \sum_{i=1}^{N} n_{i}\ln \chi _{i}
		  \]
		  \subsection{Trouton's Rules }
				   \[
						    \Delta S_{trs} = \frac{\Delta H _{trs} }{T_{trs} }
				   \]

		  \subsection{Temperature dependence}
				   \subsubsection{Constant pressure}
						    \[
								     \Delta S = S(T_{f}) - S(T_{i}) = C_{p}\int_{i}^{f} \frac{dT}{T} = C_{p}\ln \left( \frac{T_{f}}{T_{i} }  \right)
						    \]
				   \subsubsection{Constant volume}
						    \[
								     \Delta S = S(T_{f}) - S(T_{i}) = C_{v}\int_{i}^{f} \frac{dT}{T} = C_{v}\ln \left( \frac{T_{f}}{T_{i} }  \right)
						    \]
		  \subsection{Chemical reaction}
				   \begin{gather*}
						    aA + bB \to cC + dD\\
						    \Delta _{r}S^{0}  = (cS^{0} _{C} + dS^{0} _{D}) - (aS^{0} _{A} + bS^{0} _{B}  )\\
						    \Delta _{r}  S^{0} = \sum_{i=1}^{N} \nu_{i}  S^{0} _{reagents} - \sum_{i=1}^{N} \nu_{i}  S^{0} _{products} \\
				   \end{gather*}
				   If calculating \( T \neq 298 K\):
				   \begin{gather*}
						    \Delta _{r}S (T_2) = \Delta _{r}S (T_1) + \Delta C_{p}\ln \frac{T_2}{T_1}\\
						    \Delta C_{p} = (cC_{c,p} + dC_{d,p}) - (aC_{a,p} + bC_{b,p})
				   \end{gather*}
		 \section{Isothermal Transformations}
		  \subsection{Free expansion}
				   \begin{gather*}
						    \Delta T = 0 \implies \Delta U = 0 \implies q = w\\
						    w = q = - p_{ext} \Delta V = 0\\
						    \Delta H = \Delta PV \text{ (0 for perfect gas)}\\
						    \Delta S = nR\ln (\frac{V_f}{V_i}) ~ ~ ~\Delta S' = 0 ~ ~ ~ \Delta S_{tot} = \Delta S
				   \end{gather*}

		  \subsection{Expansion vs. $p_{ext}$}
				   \begin{gather*}
						    \Delta T = 0 \implies \Delta U = 0 \implies q = w\\
						    w = q = -p_{ext} \Delta V \\
						    \Delta H = \Delta PV \text{ (0 for perfect gas)}\\
						    \Delta S = nR\ln (\frac{V_f}{V_i}) ~ ~ ~\Delta S' = \frac{-q_{sistema}}{T}\\
						    \Delta S_{tot} = \Delta S + \Delta S'
				   \end{gather*}

		  \subsection{Reversible expansion}
				   \begin{gather*}
						    \Delta T = 0 \implies \Delta U = 0 \implies q = w\\
						    q = w = -nRT\ln (\frac{V_f}{V_i})\\
						    \Delta H = \Delta PV\text{ (0 for perfect gas)}\\
						    \Delta S = nR \ln (\frac{V_f}{V_i}) ~ ~ ~\Delta S' = - nR\ln (\frac{V_f}{V_i})\\
						    \Delta S_{tot} = 0
				   \end{gather*}

		 \section{Adiabatic Transformations}

		  \begin{gather*}
				   dU = dw \\
				   C_{v}dT = -pdV \\
				   w_{Adiabatic} = nC_{V,m}\Delta T\\
				   \text{For an adiabatic process is also true that:} \\
				   \gamma = \frac{C_{p,m} }{C_{V,m} } = \frac{C_{p}  }{C_{V} } \quad P_1(V_1)^{\gamma} = P_2(V_2 )^{\gamma }
		  \end{gather*}

		  \subsection{Reversible process}
				   \begin{gather*}
						    C_{V}dT = \frac{-nRT}{V}dV\\
						    \int _{T_1}^{T_2} \frac{C_{V}dT }{T}  = -nR\int _{T_1}^{T_2}\frac{dV}{V}\\
						    C_{V}\ln \left(\frac{T_1}{T_2} \right) = -R\ln \frac{V_2}{V_1} \\
						    \frac{T_2}{T_1}=\left [\frac{V_2}{V_1}\right]^{-\frac{R}{C_{V}}}
				   \end{gather*}

		  \subsection{Irreversible}
				   Take P as constant
				   \begin{gather*}
						    \int _{T_1}^{T_2} CvdT = -p \int _{V_1}^{V_2}dV\\
						    \text{Assuming } C_{V}= cost\\
						    C_{V}\Delta T = -p \Delta V
				   \end{gather*}

		  \subsection{Free expansion}

				   \begin{gather*}
						    q = 0, ~ w = -p_{ext} \Delta V = 0 \implies \Delta U = 0 \\
						    \Delta H = \\
						    \Delta S = nR\ln (\frac{V_f}{V_i}) ~ ~ ~ \Delta S' = 0 ~ ~ ~ \Delta S_{tot} = 0
				   \end{gather*}
		  \subsection{Expansion vs. $p_{ext}$}
				   \begin{gather*}
						    q = 0 \implies \Delta U = w \\
						    \Delta U = nC_{V,m} \Delta T = w = -p_{ext} \Delta V \\
						    \Delta T = -\frac{p_{ext}\Delta V}{nC_{v,m}}  \\
						    \Delta H = V \Delta p \\
						    \Delta S = nC_{V,m} \ln (\frac{T_{f}}{T_{i}}) + nR\ln (\frac{V_f}{V_{i}})\\
						    \Delta S' = 0 \implies \Delta S_{TOT}= \Delta S
				   \end{gather*}
		  \subsection{Reversible expansion}
				   \begin{gather*}
						    q = 0 \implies \Delta U = w \\
						    \Delta U = nC_{V,m} \Delta T\\
						    \Delta S = \Delta S' = \Delta _{tot} = 0
				   \end{gather*}
		 \section{Isobaric transformations}

		  \begin{gather*}
				   dp = 0  \\
				   q = \Delta H = nC_{p,m}\Delta T\\
				   w = -pdV\\
				   \Delta S = n C_{p,m} \ln (\frac{T_f}{T_i}) ~ ~ ~ \Delta S' = -nC_{p,m} \ln \frac{V_{f} }{V_{i}}\\
				   \implies \Delta S_{TOT} = 0
		  \end{gather*}

		 \section{Thermodynamic cycles}
		  \[
				   \Delta U = 0, ~ \Delta S = 0, ~ \Delta H = 0
				   .\]

		  \subsection{Carnot cycle}
				   There is 4 stage (ABCD):
				   \subsubsection{AB Reversible Isothermal expansion}
						    \begin{gather*}
								     \Delta U = 0 \implies q_{AB} = -w_{AB}\\
								     w_{AB} = -q_{AB} = -nRT_{h}\ln \left(\frac{V_{B}}{V_{A}} \right)
						    \end{gather*}

				   \subsubsection{BC Reversible Adiabatic Expansion}
						    \begin{gather*}
								     q_{BC} = 0 \implies \Delta U_{BC} = w_{BC}\\
								     w_{BC} = nC_{V,m}(T_{C}-T_{h})
						    \end{gather*}

				   \subsubsection{CD Reversible Isothermal compression}
						    \begin{gather*}
								     \Delta U = 0 \implies q_{CD} = -w_{CD}\\
								     w_{CD} = -q_{CD} = -nRT_{h}\ln \left (\frac{V_{D}}{V_{C}} \right )
						    \end{gather*}

				   \subsubsection{DA Reversible Adiabatic compression}
						    \begin{gather*}
								     q_{DA} = 0 \implies \Delta U_{DA} = w_{DA}\\
								     w_{DA} = nC_{V,m}(T_{C}-T_{h})
						    \end{gather*}

		 \section{Helmholtz's Energy}
		  \begin{gather*}
				   A = U - ST \\
				   \Delta A_{V}  = \Delta U_{V}  - T \Delta S_{V}
		  \end{gather*}
		  $\Delta A$ is the maximum (more negative) work that can be done by the system if dT=0.
		  $$ 0 \geq w \geq \Delta A $$
		  It can also be expressed with natural variables:
		  \begin{gather*}
				   dA = -SdT - pdV \\
				   dA = \left(\frac{\partial A}{\partial T}\right)_{V} dT + \left(\frac{\partial A}{\partial V}\right) _{T}dV
		  \end{gather*}

		 \section{Gibbs's Energy}
		  \begin{gather*}
				   G = H - ST\\
				   \Delta G_{p}  = \Delta H_{p}  - T\Delta S_{p}
		  \end{gather*}
		  $\Delta G$ is the maximum (more negative) work that can be done by the system if dT=0.
		  \[
				   0 \geq w_{ne} \geq \Delta G
		  \]
		  It can also be expressed with natural variables:
		  \begin{gather*}
				   dG = -SdT + Vdp\\
				   dG = \left( \frac{\partial G}{\partial T} \right)_{p}dT + \left( \frac{\partial G}{\partial p} \right)_{T}dp
		  \end{gather*}
		  \subsection{Gibbs-Helmholtz's equation}
				   \begin{gather*}
						    \frac{\partial }{\partial T} \left( \frac{\Delta G}{T} \right) _{p} = -\frac{\Delta H}{T^{2} }\\
								     \frac{\Delta G(T_2)}{T_2} - \frac{\Delta G(T_1)}{T_{1} } = \Delta H \left(
								     \frac{1}{T_2}-\frac{1}{T_1} \right)
				   \end{gather*}
		  \subsection{Pressure dependence}
				   \[
						    \Delta G = G(p_2) - \Delta G(p_1) = nRT\ln \left( \frac{p_2}{p_1} \right)
				   \]
				   It can be also used to find \( G_{m}(T_2) \) for a reaction:
				   \begin{gather*}
						    nA \to mB \\
						    nG(A,p_2) = nG^{0} + nRT\ln p_2 \\
						    nG(B,p_2) = mG^{0} + mRT \ln p_2 \\
						    \Delta G(p_2) = mG(B,p_2) - nG(A,p_2) =\\
						    \Delta G^{0} + (m-n)RT\ln p_2
				   \end{gather*}

		 \section{Chemical potential}
		  G is an extensive variable, otherwise a p, T = cost it would be always true
		  that dG = 0. G depends on the composition of the system: \( G = G(p, T, n_{1},
		  n_{2}, \ldots ) \) The partial derivative of G is the chemical potential:
		  \[
		  \mu = \left( \frac{\partial G}{\partial n_{i} } \right) _{p, T, n_{j}} \implies dG = \sum_{i=1}^{N} \mu _{i}dn_{i}
				  \]

				  \subsection{Gibbs-Duhem equation}
						   \[
								    \sum_{i=1}^{N} n_{i}d\mu _{i}= -SdT + Vdp
						   \]

						   If temperature and pressure are constant:

						   \[
								    \sum_{i=1}^{N} n_{i}d\mu _{i} = 0
						   \]
				  \subsection{Natural variables}
						   \[
								    \left( \frac{\mathrm{d\mu }}{\mathrm{dp}}   \right)_{T} = V_{m}  \quad \left( \frac{\mathrm{d\mu }}{\mathrm{dT}}  \right) _{p} = -S_{m}
						   \]
				  \subsection{Perfect gas chemical potential}
						   \begin{gather*}
								    \left( \frac{\mathrm{d\mu }}{\mathrm{dp}}  \right) _{T}\\
								    \int_{p^{0} }^{p} = \int_{p_{0} }^{p} \frac{RT}{p} = dp\\
								    \mu (p,T) = \mu _{0}(p_{0},T ) + RT\ln \left( \frac{p}{p_{0} } \right)
						   \end{gather*}
				  \section{Chemical equilibrium}
				   \( \Delta _{r}G  \) is related to G variation at constant temperature and pressure
				   \[
				   \Delta_{r} G =  \left(  \frac{\partial G}{\partial \xi} \right)_{p,T} = \sum_{i=1}^{n} \nu _{i} \mu _{i}
				  \]
				  Is possible to relate \( \Delta _{r}G^{0} \) and K :
				  \[
						   K = e ^{-\frac{\Delta _{r}G^{0}  }{RT}}
				  \]
				  Different K calculation are related easily:
				  \begin{gather*}
						   K_{c} = K_{n} \left (\frac{1}{V} \right)^{\Delta \nu}  ~ K_{\chi } = K_{n} \left(\frac{1}{n_{tot} }\right)^{\Delta \nu } \\
						   K_{p} = K_{n} \left(\frac{p_{tot}}{n_{tot}}\right) ^{\Delta \nu} \quad
				  \end{gather*}
				  \subsection{Pressure dependence}
						   \begin{gather*}
								    \frac{\partial \ln K}{\partial p} = - \frac{1}{RT} \left( \frac{\partial \Delta_{r} G}{\partial p} \right) \\ \frac{\partial \ln K}{\partial p} = - \frac{\Delta _{r}V }{RT} \\ \Delta _{r} V \approx 0 \implies \left(
										     \frac{\partial K}{\partial p} \right)_{T} = 0
						   \end{gather*}
						   Still, the equilibrium's composition is not independent of the pressure.
				  \subsection{Temperature dependence}
						   \begin{gather*}
								    \frac{\partial d\ln K}{\partial \frac{1}{T}} = - \frac{\Delta_{r}H^{0}}{R}\\ \ln \left( \frac{K_2}{K_1}
										     \right) = - \frac{\Delta_{r}H^{0}}{R}\left(\frac{1}{T_2} - \frac{1}{T_1}
										     \right)\\ K_2 = e^{\left( - \frac{\Delta_{r}H^{0} }{R}(\frac{1}{T_2} -
										     \frac{1}{T_2}) \right) } \cdot K_1
						   \end{gather*}

				  \section{Clausius-Clapeyron equation}

				   \begin{gather*}
						    \text{Clausius' equation } \quad \frac{\mathrm{dp}}{\mathrm{dT}} = \frac{\Delta S_{m}}{\Delta V_{m}}\\
						    \text{Clausius-Clapeyron} \quad \frac{\mathrm{dp}}{\mathrm{dT}} = \frac{\Delta H_{m}}{T \Delta V_{m}}
				   \end{gather*}
				   \subsection{Solid-liquid equilibrium}
						    \begin{gather*}
								     \frac{\mathrm{dp}}{\mathrm{dT}} = \frac{\Delta_{fus}  H_{m} }{\Delta V_{fus} } \frac{\mathrm{dT}}{\mathrm{T}} \\
								     p_2 = p_1 + \frac{\Delta_{fus}  H_{m} }{\Delta_{fus}  V_{m}  } \ln \left( \frac{T_2}{T_1} \right)
						    \end{gather*}
				   \subsection{Liquid-vapor equilibrium}
						    \subsubsection{Temperature dependence}
								     \begin{gather*}
										      \frac{\mathrm{dp}}{\mathrm{dT}} = \frac{\Delta _{vap}H_{m}  }{T\Delta_{vap} V_{m} }\\
										      d\ln p = \frac{\Delta_{vap}H_{m}  }{RT^{2} }dT\\
										      p_{2} = p_1 \cdot e^{\frac{\Delta _{vap}H_{m}  }{RT^{2} } \left( \frac{1}{T_2}-\frac{1}{T_1}\right)}
								     \end{gather*}
						    \subsubsection{Pressure dependence}
								     \[
										      p = p^{*} \cdot e^{\left( \frac{V_{m,liq} }{RT}\Delta p\right) }
								     \]
				   \subsection{Solid-vapor equilibrium}
						    \begin{gather*}
								     \frac{\mathrm{dp}}{\mathrm{dT}} =  \frac{\Delta _{sub}H_{m}}{T\Delta_{sub}  V_{m} }   \\
								     d\ln p = \frac{\Delta_{sub}H_{m}  }{RT^{2} }dT\\
								     p_{2} = p_1 \cdot e^{\frac{\Delta _{sub}H_{m}  }{RT^{2} } \left( \frac{1}{T_2}-\frac{1}{T_1}\right)}
						    \end{gather*}

\end{multicols}
\end{document}
