\documentclass[a4paper]{report}
\usepackage{chemfig}
\usepackage[version=4]{mhchem}
\usepackage[utf8]{inputenc}
\usepackage[T1]{fontenc}
\usepackage{textcomp}
\usepackage{babel}
\usepackage{amsmath, amssymb}
\usepackage[a4paper, margin=2cm]{geometry}
\usepackage{mhchem}
\usepackage{graphicx}
\setlength{\columnseprule}{0.4pt}
\usepackage{multicol}
\usepackage{titlesec}
\setlength\columnsep{38pt}
\usepackage{setspace} 
\usepackage{etoolbox}
\definecolor{gray75}{gray}{0.75}
%\DeclareMathSizes{12}{19}{13}{9}      % For size 11 text

\makeatletter
%\renewcommand{\@seccntformat}[1]{}
\makeatother
 
\makeatletter



\patchcmd{\l@section}
  {\hfil}
  {\leaders\hbox{\normalfont$\m@th\mkern \@dotsep mu\hbox{.}\mkern \@dotsep     mu$}\hfill}
  {}{}

\renewcommand\tableofcontents{%
    \begin{multicols}{2}[\section*{\contentsname
        \@mkboth{%
           \MakeUppercase\contentsname}{\MakeUppercase\contentsname}}]%
    \@starttoc{toc}%
    \end{multicols}%
    }

\makeatother %print dots in sections in toc.
%---------------------------------------------------------------------------%


\titleformat{\chapter}
  {\normalfont\huge\bfseries}{\thechapter}{1em}{}

\titlespacing*{\chapter}{0pt}{0pt}{20pt}
\begin{document}

\begin{titlepage}
    \begin{center}
        \vspace*{6cm}
            
        \Huge
        \textbf{FORMULAS FOR LIFE}
            
        \vspace{0.5cm}
        \LARGE
        All formulas you'll ever need
            
        \vspace{1.5cm}
            
        \textbf{Stefano Marton}
            
        \vfill
            
        A helpful tool for scientific studies   
            
        \vspace{0.8cm}
            
    \end{center}
\end{titlepage}

\pagenumbering{gobble}
\small
\setcounter{tocdepth}{2}
\tableofcontents
\normalsize 
\pagenumbering{arabic}

\chapter{Statistics}

  \begin{multicols}{2}
    \section{Mean}
      \subsection{Arithmetic mean}
      $$\overline{x}= \sum_{i=1}^n \frac{x_i}{N}$$ 
      \subsection{Geometric mean}
      $$\overline{x}=(\prod_{i=1}^n x_i)^\frac{1}{n}$$
      \subsection{Harmonic mean}
      $$\overline{x}=n(\sum^n _{i=1} \frac{1}{x_i}^{-1})$$
    \section{Standard Deviation}
      \subsection{Absolute}
      $$s=\sqrt{\frac{\sum_{i=1}^n (x_i - \overline{x})^2}{N-1}} $$
      \subsection{Relative}
      $$RSD= \frac{s}{\overline{x}}$$
  \end{multicols}

\chapter{Pure Physics}

\chapter{Pure Chemistry}

  \begin{multicols}{2}

    \section{Concentration}
        \subsection{Molarity}
        $$M=\frac{n}{V}$$

  \end{multicols}

        \subsection{Mole} 

\chapter{Thermodynamics}

  \begin{multicols}{2}

    \section{Perfect gases}
        \subsection{State equation}
      \begin{gather*}
        pV = nRT\\
        pV_{m} = RT
      \end{gather*}
        \subsection{Properties}
        \subsubsection{Molar mass}  
         $$ d_{gas} \cdot V_{m} = MW $$
        \subsubsection{Boyle's law}
        $$ p_1 V_1 = p_2 V_2 $$
        \subsubsection{Charle's law}
        $$ \frac{V_1}{T_1} = \frac{V_2}{T_2} $$
        \subsubsection{Gay-Lussac's Law}
        $$ \frac{p_1}{T_1} = \frac{p_2}{T_2} $$

    \section{Real gases}
        \subsection{Virial equation}

        \[
        pV_{m}=RT\left( 1+\frac{B}{V_{m}}+\frac{C}{V^{2}_{m}}+\dots \right) 
        \] 

        \subsection{van der Waals equation}
        \begin{gather*}
          p = \frac{nRT}{V-nb} - a \left(\frac{n}{V} \right) ^{2}  \\
          p = \frac{nRT}{V_{m} - b } - a \left(\frac{1}{V_{m} }  \right)^{2} 
        \end{gather*}

    \section{Internal Energy}
    \begin{gather*}
    dU = \delta w + \delta q \\
    dU = TdS - pdV 
    \end{gather*}
    
    \section{Entalphy}
             $$ H = U + pV $$

      \subsection{T dependence}
      \subsubsection{Pure substance}
      \[
        \Delta H(T_2) = \Delta H(T_1) + \int _{T_1} ^{T_2} C_{p}dT\\
      .\] 
    \subsubsection{Chemical reaction}
    \begin{gather*}
        \Delta (T_2) = \Delta H_{r}T_1 = \Delta C_{p}(T_2 - T_1)\\
        \Delta C_{p} = [c C_{p,C} + d C_{p,D}] - [a C_{p,A} + b C_{p,B}] \\
    \end{gather*}

    \section{Entropy} 

    \begin{gather*}
       \Delta S_{TOT} \geq 0 \\  
       \Delta S  = \frac{q}{T} \implies \Delta S = \int _{i} ^{f} \frac{q_{rev}}{T}
     \end{gather*} 
    For an ideal gas it can be simplified:
    \[
      \Delta S = nC_{m}\ln \left( \frac{T_{f}}{T_{i} }  \right) + nR\ln \left( \frac{V_{f}}{V_{i} }  \right)  
    \]

    \subsection{T dependence}
      \subsubsection{Constant pressure}
        \[
          \Delta S = S(T_{f}) - S(T_{i}) = C_{p}\int_{i}^{f} \frac{dT}{T} = C_{p}\ln \left( \frac{T_{f}}{T_{i} }  \right)  
        \] 
      \subsubsection{Constant volume} 
        \[
          \Delta S = S(T_{f}) - S(T_{i}) = C_{v}\int_{i}^{f} \frac{dT}{T} = C_{v}\ln \left( \frac{T_{f}}{T_{i} }  \right)  
        \] 
      \section{Isothermal Transformations} 
        \subsection{Free expansion}
          \begin{gather*}
            \Delta T = 0 \implies \Delta U = 0 \implies q = w\\
            w = q = - p_{ext} \Delta V = 0\\
            \Delta H = \Delta PV \text{ (0 for perfect gas)}\\
            \Delta S = nR\ln (\frac{V_f}{V_i}) ~ ~ ~\Delta S' = 0 ~ ~ ~ \Delta S_{tot} = \Delta S
          \end{gather*}

        \subsection{Expansion vs. $p_{ext}$}
          \begin{gather*}
            \Delta T = 0 \implies \Delta U = 0 \implies q = w\\
            w = q = -p_{ext} \Delta V \\
            \Delta H = \Delta PV \text{ (0 for perfect gas)}\\
            \Delta S = nR\ln (\frac{V_f}{V_i}) ~ ~ ~\Delta S' = \frac{-q_{sistema}}{T} ~ ~ ~ \Delta S_{tot} = \Delta S + \Delta S'
          \end{gather*}

        \subsection{Reversible expansion}
          \begin{gather*}
            \Delta T = 0 \implies \Delta U = 0 \implies q = w\\
            q = w = -nRT\ln (\frac{V_f}{V_i})\\
            \Delta H = \Delta PV\text{ (0 for perfect gas)}\\
            \Delta S = nR \ln (\frac{V_f}{V_i}) ~ ~ ~\Delta S' = - nR\ln (\frac{V_f}{V_i}) \implies \Delta S_{tot} = 0
          \end{gather*}

      \section{Adiabatic Transformations}

          \begin{gather*}
            dU = dw \\
            C_{v}dT = -pdV \\
            w_{Adiabatic} = nC_{V,m}\Delta T\\
            \text{For an adiabatic process is also true that:} \\
              \gamma = \frac{C_{p,m} }{C_{V,m} } = \frac{C_{p}  }{C_{V} } \quad P_1(V_1)_{\gamma } = P_2(V_2 )^{\gamma }  
          \end{gather*}

        \subsection{Reversible process}
        \begin{gather*}
          C_{V}dT = \frac{-nRT}{V}dV\\
          \int _{T_1}^{T_2} \frac{C_{V}dT }{T}  = -nR\int _{T_1}^{T_2}\frac{dV}{V}\\
          C_{V}\ln \left(\frac{T_1}{T_2} \right) = -R\ln \frac{V_2}{V_1} \\
          \frac{T_2}{T_1}=\left [\frac{V_2}{V_1}\right]^{-\frac{R}{C_{V}}}
        \end{gather*}

        \subsection{Irreversible}
        Take P as constant
        \begin{gather*}
          \int _{T_1}^{T_2} CvdT = -p \int _{V_1}^{V_2}dV\\
          \text{Assuming } C_{V}= cost\\
          C_{V}\Delta T = -p \Delta V 
        \end{gather*}

        \subsection{Free expansion}

          \begin{gather*}
            q = 0, ~ w = -p_{ext} \Delta V = 0 \implies \Delta U = 0 \\
            \Delta H = \\
            \Delta S = nR\ln (\frac{V_f}{V_i}) ~ ~ ~ \Delta S' = 0 ~ ~ ~ \Delta S_{tot} = 0
          \end{gather*}
        \subsection{Expansion vs. $p_{ext}$}
           \begin{gather*}
             q = 0 \implies \Delta U = w \\
             \Delta U = nC_{V,m} \Delta T = w = -p_{ext} \Delta V \\
             \Delta T = -\frac{p_{ext}\Delta V}{nC_{v,m}}  \\
             \Delta H = V \Delta p \\
             \Delta S = nC_{V,m} \ln (\frac{T_{f}}{T_{i}}) + nR\ln (\frac{V_f}{V_{i}})\\
             \Delta S' = 0 \implies \Delta S_{TOT}= \Delta S 
         \end{gather*}
        \subsection{Reversible expansion}
          \begin{gather*}
            q = 0 \implies \Delta U = w \\
            \Delta U = nC_{V,m} \Delta T\\
            \Delta S = \Delta S' = \Delta _{tot} = 0
          \end{gather*}
      \section{Isobaric transformations}
      \subsection{Reversible}
        \begin{gather*}
          q = \Delta H = nC_{p,m}\Delta T\\
          w = -pdV\\
          \Delta S = n C_{p,m} \ln (\frac{T_f}{T_i}) ~ ~ ~ \Delta S' = -nC_{p,m} \ln \frac{V_{f} }{V_{i}}\\
          \implies \Delta S_{TOT} = 0 
        \end{gather*} 

    \section{Thermodynamic cycles}
    \[
      \Delta U = 0, ~ \Delta S = 0
    .\] 

      \subsection{Carnot cycle}
      There is 4 stage (ABCD):
        \subsubsection{AB Reversible Isothermal expansion}
        \begin{gather*}
          \Delta U = 0 \implies q_{AB} = -w_{AB}\\
          w_{AB} = -q_{AB} = -nRT_{h}\ln \left(\frac{V_{B}}{V_{A}} \right)
        \end{gather*}
        \subsubsection{BC Reversible Adiabatic Expansion}
        \begin{gather*}
          q_{BC} = 0 \implies \Delta U_{BC} = w_{BC}\\
          w_{BC} = nC_{V,m}(T_{C}-T_{h})  
        \end{gather*}
        \subsubsection{CD Reversible Isothermal compression}
        \begin{gather*}
          \Delta U = 0 \implies q_{CD} = -w_{CD}\\
          w_{CD} = -q_{CD} = -nRT_{h}\ln \left (\frac{V_{D}}{V_{C}} \right )
        \end{gather*}
        \subsubsection{DA Reversible Adiabatic compression}
        \begin{gather*}
          q_{DA} = 0 \implies \Delta U_{DA} = w_{DA}\\
          w_{DA} = nC_{V,m}(T_{C}-T_{h})  
        \end{gather*}

        
        
      

  \end{multicols}

\chapter{Radiations} 

  \begin{multicols}{2}
  
    \section{Bragg Equation}

      $$n \lambda = 2d_{hkl} \sin \theta$$

 \end{multicols}

\end{document} 

